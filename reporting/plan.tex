\documentclass{article}
\usepackage{graphicx}
\usepackage{hyperref}

\begin{document}
\title{Plan for the analysis of Kaggle Titanic survival data and creation of a software to predict survival}
\author{Tuomo Kalliokoski}

\maketitle

\begin{abstract}
This is an example data analysis and machine learning project for showcasing quality of my work. The project will produce material for publication in social media and others beside the competition result. This is to be archieved with use of Jupyter and Python as the main tools.
\end{abstract}

\section{Project deliverables}
The goal for this project is creation of following items: 
\begin{enumerate}
	\item An entry to the Getting strated Prediction Competition at Kaggle
	\item A plan of action (this file)
	\item An executive summary of the project
	\item A Detailed report of the project
	\item A presentation of the results
	\item A Jupyter notebook file of the preliminary analysis of the data
	\item Separate Jupyter notebooks of the actual prediction models and their results
	\item A production code for making survival predictions
\end{enumerate}

\subsection{An entry to the Getting Started Prediction Competition}
Kaggle has competitions for Machine Learning projects and this is their Getting Started Prediction Competition. From three provided files: train.csv, test.csv, and gender\_submission.csv one has to make a machine learning model using information from the train.csv to provide prediction for test.csv with the prediction being in format of gender\_submission.csv.
 
\subsection{A plan of action}
This file, used as a reference for the actual work to be done.

\subsection{An executive summary of the project}
The summary of the results of the analysis. A short, maximum length of 2 pages, report of the results focusing on the actual findings. Target audience: non-techincal.  Format: pdf generated from latex-files.

\subsection{A detailed report of the project}
A detailed report with findings from the data and comparison between analysis methods and tools. Style: a scientific study. Target audience: peers.  Format: pdf generated from latex-files.

\subsection{A presentation of the results}
Most likely slide show with presentation notes. This includes the comparison between methods. Target audience: non-technical. Format: most likely pptx.

\subsection{A Jupyter notebook file of the preliminary analysis of the data}
Jupyter notebook file with proper markup documentation for later use storing the preliminary analysis and its results.

\subsection{Separate Jupyter notebooks of the actual prediction models and their results}
 Jupyter notebook files with proper markup documentation for later use storing the history and development of the actual prediction codes for their respective prediction technologies.

\subsection{A production quality code for making survival predictions}
Python program for generating survival predictions.

\section{The Plan of action}
\begin{enumerate}
	\item Produce the plan of action
	\item Get the data
	\item Setup basic system for the analysis
	\item Perform preliminary data analysis
	\item Select the prediction machine technologies
	\item Produce the predictions
	\item Generate the production code
	\item Create reports
	\item Publish results
\end{enumerate}

\subsection{Produce the plan of action}
This is it.

\subsection{Get the data}
Data is freely available for download from the \href{https://www.kaggle.com/c/titanic/overview}{Kaggle website}.

\subsection{Setup basic system for the analysis}
An Anaconda environment with Jupyter Lab interface for notebooks and their creation, Spyder for production code generation are to be created. Git and Github repository are to be set up for the project.

\subsection{Perform preliminary data analysis}
A regular preliminary analysis for missing values, correlations, and other behavior of the data. A goal is to produce graphical results for reports. Proper coding practices including documentation and usage of version control is paramount.

\subsection{Select the prediction machine technologies}
Different style of data requires different technologies for the prediction machine. At the moment following three technologies are to be investigated: a decision tree, graph clustering and an artifician neural network. Preliminary analysis may change this approach.

\subsection{Produce the predictions}
Main goal in this is to produce predictions with jupyter notebook, but proper analysis of the prediction accuracy is vital. For reporting purposes the efficiency of the used methology is also important as is comparison between parameter choise for the methods used. Proper coding practises including documentation and version control is paramount.

\subsection{Generate the production code}
Generation and testing of a software mentioned in goals. This software is required to follow good coding practises including documentation.

\subsection{Create reports}
Writing of the reports, presentations and others mentioned in goals.
 
\subsection{Publish results}
Submiting the entry to the Kaggle competition, all appropriate files are to be published in git, blog posting and linkedin posting are to be done.

\section{Schedule}
Preliminary schedule for project
\begin{itemize}
	\item The plan of action, data download and setup by 2020-07-24 T18:00+03:00.
	\item The preliminary data analysis and selection of techniques by 2020-07-26T18:00+03:00.
	\item The actual prediction production by 2020-07-31T18:00+03:00
	\item The production code finalized by 2020-08-02T18:00+03:00
	\item The reports by 2020-08-07T18:00+03:00
	\item The publication by 2020-08-09T18:00+03:00
\end{itemize}


\end{document}